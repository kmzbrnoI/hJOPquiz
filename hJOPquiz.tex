\documentclass[12pt,a4paper]{article}
\usepackage{xltxtra}
\usepackage{polyglossia}
\setdefaultlanguage{czech}
\defaultfontfeatures{Mapping=tex-text}
\setmainfont{Carlito} % Cambria on linux
% \setmonofont{Consolas} % Uncommon font for linux
\def\uv#1{„#1“}

\usepackage[ unicode
           , pdfauthor={Jan Horáček}
           , pdftitle={hJOPquiz},
           , pdfsubject={Kvíz testující schopnost řídit hJOP},
           , plainpages=false
           , pdfpagelabels
           , draft=false
           , colorlinks=false
           , unicode=true
           ]{hyperref}
\usepackage{graphicx}

\usepackage{microtype}
\usepackage{enumitem}
\usepackage{titling}
\usepackage{xcolor}
\textwidth 16cm \textheight 24.6cm
\topmargin -1.3cm
\oddsidemargin 0cm

\def\solution#1{\ifsolution \par\vskip-\parskip{\color{gray}#1}\fi}

\def\symbol#1#2#3{ % filename, scale, description
    \ifsolution
       \parbox[t]{.305\linewidth}{\parskip 0pt\centering\includegraphics[width=#2\textwidth]{#1}\par
       \vskip -5pt\vrule height 0pt depth 3\baselineskip width 0pt\vtop{\color{gray}
                #3\vfill}}\quad
    \else
       \includegraphics[width=#2\textwidth]{#1}%
    \fi
}


\begin{document}
\thispagestyle{empty}

\setlength{\parindent}{0cm}
\setlength{\parskip}{.4\baselineskip plus2pt minus1pt}
\setlength{\droptitle}{-5em}
\setlist[enumerate]{itemsep=0pt}

\title{\bfseries
{\Large Ovládání kolejiště pomocí hJOP\\}
{\LARGE Podklady pro zkoušku oprávnění řízení\\}
\ifsolution
{\Large \color{gray}včetně řešení\\}
\fi
{\small v2.2}}
\author{Jan Horáček (jan.horacek@kmz-brno.cz)}
\date{\today}
\maketitle

Tento dokument obsahuje podklady pro učení a testování způsobilosti řízení
hJOP. Jeho cílem je nastavit jednotnou míru znalostí obsluhy kolejiště, a~tím
předejít zbytečným chybám, zdržování provozu, újmám na majetku a~přispět
k~obecně lepší součinnosti mezi obsluhou kolejiště.

Následující text obsahuje několik úrovní zaškolení – od základní až po tu
nejpokročilejší. Každá úroveň obsahuje teoretické otázky a~praktickou část, ve
které probíhá řešení zadaných úkolů přímo na kolejišti. Každý, kdo chce získat
příslušné oprávnění, musí odpovědět na všechny otázky a~být schopen vyřešit
všechny praktické problémy. Otázky v~jednotlivých sekcích jsou seřazeny od
nejdůležitějších k~méně důležitým.

\section{Doporučená metodika zaškolování}

\begin{enumerate}[leftmargin=*]
\item Nováčka zařadit k~někomu (patronovi) na úrovni, kterou požaduje, ten ho
naučí základy obsluhy.
\item Nováčkovi dát k~dispozici tento dokument, on si přečte dotazy, případné
nejasnosti vyřeší se zaškoleným patronem.
\item Oprávnění získává nováček výhradně u~školitele – u~kohokoliv s~oprávněním
A.
Školitel vybere otázky z~tohoto dokumentu a~nechá nováčka odpovědět. Školitel
provede s~no\-váč\-kem praktickou část zkoušky.
\item Školitel přidává login nováčka do hJOPserveru, nováček je oprávněn řídit.
\end{enumerate}

\newpage

\section{S0 – strojvedoucí hJOPdriver}

Opravňuje k~řízení vlaků a posunových dílů přes mobilní aplikaci hJOPdriver.

\subsection{Praktická část}

\begin{enumerate}[leftmargin=*]
\item Proveďte převzetí vlaku na ruční řízení, popište jednotlivé kroky přebírání.
\item Řiďte vlak z~jedné stanice do druhé. Dbejte na řádné sjednání jízdy
s~výpravčím.
\item Uvolněte vlak z~aplikace.
\item Demonstrujte řízení lokomotivy v~režimu \textit{posun}. Dbejte na řádné
sjednání posunu s~výpravčím.
\solution{Při sjednání posunu výpravčí musí strojvedoucímu sdělit o~jaký
posunový díl se jedná a kam se tento díl smí posouvat. Příklad (objetí soupravy): \\
V: \uv{Modrý hektor na 1. koleji posun do záhlaví směr Ivančice povolen.} \\
S: \uv{Hektor je v záhlaví.} \\
V: \uv{Hektor přes 3. kolej do záhlaví směr Čejč posun povolen.} \\
S: \uv{Hektor je v záhlaví směr Čejč.} \\
V: \uv{Hektor na 1. obsazneou kolej posun povolen, připoj se k~soupravě.}
}

\item Demonstrujte poznání trati, ukažte polohy jednotlivých návěstidel.
\item Proveďte nouzové zastavení vlaku a~opětovné rozjetí.
\item Převezměte vlak pouze v~režimu ovládání funkcí a řiďte takto vlak na trati.
\item Popište významy vybraných návěstí na návěstním opakovači a doporučení chování
pro strojvedoucího (např. zrychlit, zpomalit apod.).
\item Proveďte posun s~obsluhou pomocného stavědla dle instrukcí výpravčího.
\item Demonstrujte řízení lokomotiv v~multitrakci.
\end{enumerate}

\subsection{Teoretická část}
\begin{enumerate}[leftmargin=*]
\item V jakých základních režimech lze ručně řídit hnací vozidla? Jaké jsou rozdíly
těchto režimů?
\solution{Jako vlak nebo jako posun. Rozdílů je hodně – způsob sjednávání jízdy,
ne/funkčnost návěstního opakovače, maximální rychlost. Je třeba znát specifika
obou režimů.}

\item Popište proces ručního řízení lokomotivy v~režimu \textit{posun} od zapnutí
aplikace až do ukončení posunu. Na co je třeba si během posunu dávat pozor?
\solution{Nejprve žádáte výpravčího o~lokomotivu. Sjednáte si s ním, kam má
posunový díl jet a co má dělat. Dáváte si pozor, abyste nevyjeli za hranice
stanice (za bílý označník s modrou čepičkou). Posun vždy řídí výpravčí, není
možné posunovat bez sjednání posunu s~ním. Výpravčí určuje, kdy a kam se bude
sunout, má právo posun kdykoliv přerušit. Při posunu se nedbá na návěst na
návěstním opakovači. Svévolné ježdění s posunovým dílem bude mít za následek
odebrání práva k ručnímu řízení! Posun končí uvolněním lokomotivy z~ručního řízení.}

\item K~čemu slouží v~ovladači volba \textit{Ruční řízení}?
\solution{Při nezaškrtnutém \textit{Ručním řízení} řídíte pouze funkce, nikoliv
jízdu. Po zaškrtnutí \textit{Ruční řízení} řídíte i~jízdu vlaku.}

\item Kdy řídí vlak strojvedoucí a kdy počítač? Kdy přechází odpovědnost za
řízení vlaku z~počítače na strojvedoucího a zpět?
\solution{Pokud je zaškrtnuto \textit{Ruční řízení}, řídíte jízdu vlaku
výhradně vy, jízdní povely počítače se ignorují. Pokud je volba \textit{Ruční řízení}
odškrtnutá, jízdu řídí počítač, funkce však řídíte vy. Při zaškrtnuté volbě
\textit{Ruční řízení} máte za vlak plnou odpovědnost včetně případné úhrady
a~oprav škod způsobených vaši nedbalostí.}

\item Kdy pouští zvuky vlaku automaticky počítač a kdy ručně strojvedoucí?
\solution{Kdykoliv je hnací vozidlo vlaku převzato v aplikaci, a~to i~bez
zaškrtnutí volby \textit{ruční řízení}, zvuky pouští výhradně strojvedoucí.}

\item Jak strojvedoucí pozná, kde s~vlakem zastavit a kam nesmí jet?
\solution{Strojvedoucí musí během jízdy kontrolovat návěsti na návěstním
opakovači v~aplikaci a řídit se jimi. Strojvedoucí nesmí vlakem projet
návěstidlo s~návěstí zakazující jízdu. Strojvedoucí musí zastavit, jakmile je
maximální povolená rychlost vlaku 0 km/h.}

\item Jaká je maximální rychlost pro posun?
\solution{40 km/h. Při sunutí vozů nejvýše 30 km/h. Při sunutí je třeba jet
obzvláště opatrně a~pomalu, vagóny jsou oproti lokomotivě lehčí a proto můžou
snadno vyskočit na výhybkách.}

\item Jaká je maximální rychlost vlaku?
\solution{Momentální dovolená maximální rychlost se zobrazuje pod návěstním
opakovačem v~aplikaci. Strojvedoucí nesmí tuto rychlost překročit.}

\item Kdo odpovídá za dodržování maximální rychlosti?
\solution{Aplikace nijak nekontroluje překročení maximální rychlosti.
Strojvedoucí musí její nepřekročení hlídat sám.}

\item Proč se hnací vozidlo nerozjede ihned po posunutí posuvníku rovnou na
požadovanou rychlost?
\solution{Každé hnací vozidlo má tzv. rozjezdovou křivku, díky které se rozjíždí
pomalu. Při rozjezdu je tedy třeba počkat na ustálení rychlosti a ne posuvník
hned posouvat na co nejvyšší hodnoty.}

\item Na co je třeba dávat pozor při brzdění?
\solution{Každé hnací vozidlo má tzv. brzdnou křivku, která způsobuje, že i~po
posunutí rychlostního posuvníku na menší rychlost vozidlo dobržďuje pomalu.
Jedná se řádově o~50 cm pro zastavení z~rychlosti 40 km/h, tedy poměrně pomalé
brzdění! \textbf{Proto je třeba začít brzdit před místem zastavení včas!}}

\item Jaká pravidla platí pro zastavení osobních vlaků ve stanicích?
\solution{Vlaky zastavují u~perónů a to nejblíže dopravní kanceláři nebo
podchodu, aby mohli lidé pohodlně odcházet. Při křižování jsou možné 2 scénáře:
\uv{čely k sobě} nebo \uv{konci k~sobě}.}

\item Jaká specifika platí pro řízení jízdy vlaku s~více lokomotivami v~multitrakci?
\solution{U~všech lokomotiv, které chcete řídit v~multitrakci, musíte
zaškrtnout \textit{Multitrakce}. Funkce se řídí zvlášť, směr se řídí zvlášť,
rychlost se řídí dohromady. Po dokončení řízení musíte uvolnit všechny
lokomotivy.}

\item Lze mobilní telefon půjčit někomu jinému?
\solution{Půjčovat mobilní telefon můžou pouze zkušení strojvedoucí s~hodinami
praxe. Ten, kdo mobilní telefon půjčí, musí na něj pořád dávat pozor a
kontrolovat, že se host chová bezpečně. Zodpovědnost za řízení vlaku má stále
majitel telefonu!}

\item Co může být důvodem zrušení přístupu k~ručnímu řízení?
\solution{Důvodem je závažné porušení pravidel popsaných v tomto dokumentu a~ve
školení – například neoprávněné projetí stůj, posun bez sjednání s~výpravčím nebo
překročení maximální rychlosti. Odebrání přístupu může být dočasné nebo trvalé.}

\item Popište, jak funguje jízda na přivolávací návěst.
\solution{Přivolávací návěst (PN) se používá k~povolení jízdy vlaku
v~situacích, kdy došlo k~poruše zabezpečovacího zařízení nebo k~nestandardní
situaci (např.  vjezd na obsazenou kolej). Při jízdě na PN musí jet
strojvedoucí nejvýše 40 km/h a~musí být schopen včas zastavit před libovolnou
překážkou. Přivolávací návěst se typicky nepřenáší na návěstní opakovač, takže
strojvedoucí musí sledovat fyzické návěstidlo, reliéf nebo se domluvit
s~výpravčím.}

\end{enumerate}

\newpage

\section{S1 – dispečer základní}

Opravňuje k~řízení menších stanic.

Pro tuto úroveň zaškolení je doporučeno mít zaškolení na úrovni S0. Výjimkou
je situace, kdy dispečeři používají k~řízení jízdy jiné prostředky než mobilní
aplikaci hJOPdriver.

\subsection{Praktická část}

\subsubsection*{Přihlášení, odhlášení}
\begin{enumerate}[leftmargin=*]
\item Spusťte panel. Připojte se k~serveru, přihlaste se, odpojte se od
serveru. Přihlaste se v~režimu čtenáře, demonstrujte přihlášení více panelů.
\item Proveďte přihlášení k~jednomu panelu a ke všem spuštěným panelům.
\item Zobrazte a skryjte popisky bloků.
\end{enumerate}

\subsubsection*{Základní obsluha vlaků}
\begin{enumerate}[leftmargin=*]
\item Obslužte několik vlaků.
\item Proveďte křižování osobních vlaků.
\end{enumerate}

\subsubsection*{Dopravní kancelář}
\begin{enumerate}[leftmargin=*]
\item Demonstrujte komunikaci se sousedními dispečery (telefon, poslání zprávy).
\item Demonstrujte přepnutí řízení na místní a dálkový provoz.
\end{enumerate}

\subsubsection*{Soupravy}
\begin{enumerate}[leftmargin=*]
\item Proveďte vytvoření soupravy.
\item Proveďte smazání soupravy.
\item Zobrazte seznam souprav v~obvodech všech vámi řízených stanic, vysvětlete
význam tohoto seznamu.
\item Upravte výchozí a cílovou stanici soupravy v~koncové stanici.
\end{enumerate}

\subsubsection*{Bloky}
\begin{enumerate}[leftmargin=*]
\item Proveďte ruční přestavení výhybky.
\item Proveďte nouzové ruční přestavení výhybky.
\item Vyřešte uvázlý vagón v~trati.
\item Proveďte nouzové uvolnění závěrů úseků. Vysvětlete, kdy se takové uvolnění
provádí.
\item Demonstrujte obsluhu rozpojovačů.
\item Zaveďte štítek na úseku.
\end{enumerate}

\subsubsection*{Jízdní cesty}
\begin{enumerate}[leftmargin=*]
\item Proveďte zrušení JC. Ukažte co nejvíce postupů.
% \solution{(a) Volba \texttt{RC} na návěstidle, u~které začíná jízdní cesta
% (JC). (b) Postupným nouzovým uvolněním závěrů (\texttt{NUZ}) všech úseků JC.
% Závěr staniční koleje se uvolňuje automaticky (neoznačuje se volbou \texttt{NUZ}).}

\item Demonstrujte posun ve stanici.

\item Proveďte obsluhu vlaku v~ručním režimu řízení, včetně stavění jízdních
cest.
\end{enumerate}

\subsubsection*{Ruční řízení}
\begin{enumerate}[leftmargin=*]
\item Předveďte ruční řízení vlaku přes aplikaci Jerry.
\item Předveďte komunikaci se strojvedoucím.
\item Demonstrujte násilné převzetí soupravy od strojvedoucího.
\end{enumerate}

\subsubsection*{Krizové scénáře}
\begin{enumerate}[leftmargin=*]
\item Proveďte nouzové zastavení provozu na celém kolejišti.
\item Proveďte nouzové zastavení jedné lokomotivy.
\item Vyřešte situaci, kdy vlak přejel do následujícího kolejového obvodu.
\item Vyřešte situaci, kdy vlak rozřezal výhybku a způsobil zkrat.
\end{enumerate}

\subsubsection*{Poznání}
\begin{enumerate}[leftmargin=*]
\item Demonstrujte poznání trati, ukažte polohy jednotlivých návěstidel, typy
traťových zabezpečovacích zařízení.
\end{enumerate}


\subsection{Teoretická část}

\subsubsection*{Přihlašování, odhlašování}
\begin{enumerate}[leftmargin=*]

\item Vysvětlete, jak funguje mechanismus přihlašování k~více panelům zároveň.
Kdy tento mechanismus použít a~kdy naopak ne?
\solution{Pro přihlášení k~více panelům zároveň je nutné nejdříve spustit
všechny požadované panely, pak se v~jednom z~nich připojit k~serveru a~ponechat
zatrhnutou možnost \textit{Autorizovat další panely}. Tím se automaticky
přihlásí všechny další spuštěné panely. \\ Volbu \textit{Autorizovat další
spuštěné panely} je vhodné zrušit v~případě, kdy se chcete přihlásit
pouze k~jedné stanici, například jako čtenář. Pokud byste tuto volbu ponechali
zatrhnutou, všechny ostatní spuštěné panely, na které jste přihlášeni vy,
se přehlásí na čtenáře, a~vy tak ztratíte možnost ovládat kolejiště.}

\item Jak poznáte, že je panel připojen k~serveru?
\solution{Vlevo dole na panelu je napsáno \textit{Připojeno k~serveru}}.

\item Vysvětlete význam všech barev symbolu dopravní kanceláře. Co je to místní
a~dálkový provoz?
\solution{Místní provoz (šedý symbol DK) = ovládáte stanici, ovládáte ji pouze
vy a~nikdo jiný. Dálkový provoz (bílý symbol DK) = pozorujete stanici,
neovládáte, stanici nikdo neovládá. Červený symbol DK = stanici ovládá někdo
jiný.}

\item Popište postup přebírání stanice jiným výpravčím, například při změně
směn.
\solution{Odcházející výpravčí se odhlásí, nový výpravčí se přihlásí na svůj
účet.}

\item Jako výpravčí si potřebujete odskočit (například na toaletu). Co musíte
provést?
\solution{V této situaci máte několik možností. Vždy je dobré váš záměr
oznámit výpravčímu sousední stanice, nejlépe mu na dobu vaši nepřítomnosti
předat řízení vaši stanice. Pokud sousední výpravčí není dostupný, nemá čas nebo
řízení stanice nechce, je před odchodem důležité se z~panelu odhlásit. A~to nejenom
přepnutím řízení na dálkový provoz, ale i~odhlášením uživatele. Pokud byste se
pouze přepnuli na dálkový provoz, mohl by přijít návštěvník a~začít stanici
řídit kolejiště, což nechceme. \\
Obecně byste po dobu své služby měli být u~své stanice a~odcházet jen ve
výjimečných situacích.}

\item Chcete se přihlásit ke stanici pouze na dívání, jak to provedete?
\solution{Při přihlašování zvolíte \textit{Přihlásit se jako host} a ideálně
odškrtnete \textit{Autorizovat další panely}.}

\item Jak potlačíte zvukovou notifikaci zkratu na kolejišti? Je toto
potlačení trvalé?
\solution{Na horní liště je ikona s~reproduktorem a křížkem. Potlačení trvá
1~minutu.}

\item Můžete z~libovolného pracoviště ovládat libovolnou stanici?
\solution{Pokud je ikona stanice dostupná na ploše, ano.}

\item Jakými všemi metodami můžete komunikovat s~výpravčími sousedních stanic?
\solution{Napsání zprávy, zavolání telefonem, rozhovor osobně. Při komunikaci
na sebe zásadně nepořváváme, používáme výše uvedené prostředky.}

\item Ke kolejišti přijde váš kamarád, který si chce zajezdit, ale nemá
zaškolení (nemá login), co mu můžete povolit a co naopak nesmíte? Jak se k~němu
máte chovat?
\solution{Kamarád nesmí samostatně řídit stanici. Můžete ho pustit k~počítači
a~nechat ho provádět úkony výpravčího, ale musí být pod vaším trvalým dohledem.}

\end{enumerate}

\subsubsection*{Soupravy}
\begin{enumerate}[leftmargin=*]
\item Jaký je rozdíl mezi vlakem, lokomotivou a~soupravou? Může se lokomotiva
sama pohybovat na širé trati?
\solution{Vlak = souprava. Souprava je složena z~jedné nebo více lokomotiv.
Lokomotiva se na trati typicky pohybuje jako lokomotivní vlak.}

\item Jak poznáte typ blížícího se vlaku? Jaká rozhodnutí na základě typu
typicky děláte?
\solution{Z~předčíslí soupravy. Podle typu vlaku například volíte kolej, na
kterou vlak přijmete: osobní vlaky vždy k~perónu.}

\item Vyjmenujte, jaká předčíslí souprav mají jaký význam.
\solution{Viz nápovědu v~\texttt{EDIT vlak} okně hJOPpanelu.}

\item Jaké je typické složení čísla vlaku? Jak poznat z~čísla vlaku DCC adresu
lokomotivy?
\solution{Číslo vlaku je typicky šestimístné číslo složené z~dvojmístného
předčíslí (typ soupravy) a~čtyřmístné adresy dekodéru hnacího vozidla. Pokud má
souprava více hnacích vozidel, jsou poslední 4 cifry čísla soupravy na volbě
výpravčího.}

\item Vysvětlete rozdíl mezi možným směrem vlaku a~směrem stanoviště A.
\solution{Směr vlaku udává možný směr, kterým se vlak může pohybovat. Může být
jeden nebo i~oba směry. Podle možného směru vlaku se vykresluje šipka nad
číslem vlaku v~reliéfu. Možný směr vlaku neovlivňuje fyzický směr jízdy
lokomotivy. Možný směr vlaku je vlastnost vlaku. \\ Orientace stanoviště A~je
vlastnost lokomotivy. Udává, jakým směrem je lokomotiva fyzicky otočena na
kolejišti. Je nutné jej správně zadat, aby lokomotiva jela správným směrem.}

\item Jak poznat, že mají hnací vozidla vlaku správně zadána stanoviště A?
\solution{Všechny lokomotivy vlaku svítí správným směrem.}

\item Kdy je a~kdy není třeba upravovat výchozí a~cílovou stanici soupravy?
\solution{Výchozí a~cílová stanice se automaticky mění ve smyčkách. Pokud vlak
končí v~jiné stanici, je třeba mu upravit výchozí a~cílovou stanici ručně.
Vlak, který dorazit do cílové stanice, má své číslo šedě podbarvené.}

\item Kolik lokomotiv může mít souprava?
\solution{Až 4.}

\item Je nutné měnit orientaci stanoviště A~při otáčení vlaku v~koncové
stanici?
\solution{Ne. Naopak, dělat se to nesmí!}

\item Je nutné zadávat vlaku jeho délku a~typ? Proč?
\solution{Ano. Pro správné zastavování v~zastávkách a~u~perónů ve stanicích.}

\item Na které místo ve vlaku je možné přivěsit čisticí vůz?
\solution{Výhradně hned za lokomotivu.}

\end{enumerate}

\subsubsection*{Bloky}
\begin{enumerate}[leftmargin=*]
\item Jak poznáte zkrat na kolejišti? Co s~ním dělat?
\solution{Fialové podbarvení úseků. Zkrat je typicky způsoben najetím do špatně
přestavené výhybky: ručně zastavte lokomotivu a~na panelu nouzově přestavte
výhybku.}

\item Vysvětlete význam následujících symbolů na reliéfu. \\
\symbol{symboly/kol1.png}{0.1}{Kolej s~prostředky pro kontrolu volnosti volná.}
\symbol{symboly/kol2.png}{0.1}{Kolej obsazená.}
\symbol{symboly/kol3.png}{0.1}{Kolej volná, závěr vlakové cesty.}
\symbol{symboly/kol4.png}{0.1}{Kolej volná, závěr posunové cesty.}
\symbol{symboly/kol22.png}{0.1}{Kolej volná, zkrat.}
\symbol{symboly/kol8.png}{0.1}{Kolej volná, výpadek DCC.}

\symbol{symboly/hlnav16.png}{0.1}{Hlavní návěstidlo nekomunikuje.}
\symbol{symboly/hlnav1.png}{0.1}{Základní stav.}
\symbol{symboly/hlnav2.png}{0.1}{Povolující návěst pro vlak (mimo PN).}
\symbol{symboly/hlnav3.png}{0.1}{Povolující návěst pro posun. Přerušovaně:
přivolávací návěst.}
\symbol{symboly/hlnav5.png}{0.1}{Návěstidlo mění návěst / návěstidlo zhaslé.}
\symbol{symboly/hlnav6.png}{0.1}{Zamknuto do návěsti stůj.}
\symbol{symboly/hlnav8.png}{0.1}{Základní stav + volba vlakové cesty.}

\symbol{symboly/senav6.png}{0.1}{Seřaďovací návěstidlo nekomunikuje.}
\symbol{symboly/senav2.png}{0.1}{Povolující návěst.}
\symbol{symboly/senav3.png}{0.1}{Návěstidlo mění návěst / návěstidlo zhaslé.}
\symbol{symboly/senav4.png}{0.1}{Zamknuto do návěsti posun zakázán.}
\symbol{symboly/senav8.png}{0.1}{Základní stav + volba posunové cesty.}

\symbol{symboly/vyh4.png}{0.1}{Ztráta komunikace.}
\symbol{symboly/vyh1.png}{0.1}{Přímý směr + úsek volný.}
\symbol{symboly/vyh2.png}{0.1}{Odbočný směr + úsek volný.}
\symbol{symboly/vyh3.png}{0.1}{Úsek volný + ztráta dohledu.}
\symbol{symboly/vyh5.png}{0.1}{Úsek obsazen.}
\symbol{symboly/vyh6.png}{0.1}{Úsek obsazen + ztráta dohledu.}
\symbol{symboly/vyh28.png}{0.1}{Výpadek napájení zesilovače (napěťová výluka) +
závěr vlakové cesty.}
\symbol{symboly/vyh37.png}{0.1}{Výhybka nevybavená zařízením pro kontrolu polohy.}
\symbol{symboly/vyh38.png}{0.1}{S~kontrolou volnosti kolejového úseku + úsek obsazen.}
\symbol{symboly/vyh39.png}{0.1}{Bez kontroly volnosti kolejového úseku.}

\symbol{symboly/vyk19.png}{0.1}{Ztráta komunikace.}
\symbol{symboly/vyk1.png}{0.1}{Na koleji + úsek volný.}
\symbol{symboly/vyk2.png}{0.1}{Sklopená + úsek volný.}
\symbol{symboly/vyk3.png}{0.1}{Na koleji + úsek obsazen.}
\symbol{symboly/vyk4.png}{0.1}{Sklopená + úsek obsazen.}
\symbol{symboly/vyk5.png}{0.1}{Ztráta dohledu + úsek volný.}
\symbol{symboly/vyk6.png}{0.1}{Ztráta dohledu + úsek obsazen.}

\symbol{symboly/ez6.png}{0.05}{Ztráta komunikace.}
\symbol{symboly/ez1.png}{0.05}{Klíč zapevněn.}
\symbol{symboly/ez3.png}{0.05}{Klíč vyjmut.}
\symbol{symboly/ez5.png}{0.05}{Ztráta kontroly.}

\symbol{symboly/prej1.png}{0.1}{Otevřen.}
\symbol{symboly/prej2.png}{0.1}{Uzavřen povelem dispečera.}
\symbol{symboly/prej3.png}{0.1}{Nouzově otevřen.}

\item Kdy je a~kdy není nutné žádat o~traťový souhlas?
\solution{O~traťový souhlas není nutné žádat, pokud ho máte udělen, nebo pokud
jste přihlášený do obou stanic tratě. V~ostatních případech žádat musíte.}

\item Co je to závěr a k~čemu slouží?
\solution{Závěr je zapevnění bloku proti změně. Typicky například zamknutí
výhybky. Závěr se typicky uděluje na bloky v~jízdní cestě. Na výhybky,
výkolejky, úvazky, zámky a pomocná stavědla lze ručně udělit nouzový závěr
volbou \texttt{ZAV>} v~menu bloku.}

\item Sousední stanice vás žádá o~traťový souhlas, vy ho však chcete přijmout až
za minutu, jak nejlépe sdělit sousední stanici, že má počkat?
\solution{Napsat zprávu, zavolat, případně nechat žádost minutu pípat.
Zvuk žádosti lze dočasně potlačit klikem na ikonu v~horním panelu.}

\item Jak poznáte dopravní a~manipulační kolej?
\solution{Manipulační kolej je ohraničena seřaďovacími návěstidly. Pozor:
přerušovaný symbol koleje na reliéfu není manipulační kolej! Přerušovaný symbol
koleje na reliéfu je kolej bez indikace obsazení.}

\item Jak poznáte směr trati na reliéfu?
\solution{Ze směru šipky úvazky.}

\item Úvazka je červená, co to znamená?
\solution{Nastala porucha blokové podmínky trati.}

\end{enumerate}

\subsubsection*{Jízdní cesty}
\begin{enumerate}[leftmargin=*]
\item Má mít vlak v~ručním řízení postavenou jízdní cestu pro jízdu na
kolejišti?
\solution{Ano, vždy musí mít.}

\item Popište vztah mezi pojmy \textit{jízdní cesta}, \textit{vlaková cesta}
a~\textit{posunová cesta}.
\solution{Jízdní cestou rozumíme cestu pro vlak (vlaková cesta) nebo pro posun
(posunová cesta).}

\end{enumerate}

\subsubsection*{Ruční řízení}
\begin{enumerate}[leftmargin=*]
\item Vyjmenujte všechny možnosti, jak řídit lokomotivu ručně.
\solution{(a) Skrze aplikaci Jerry. (b) Skrze aplikaci hJOPdriver. (c) Pomocí
Roco Multimaus a~uLI.}

\item Co znamená, když je v~ovladači zaškrtnuto \textit{ruční řízení}
a k~čemu vás to opravňuje?
\solution{\textit{Ruční řízení} znamená, že řídíte jízdu vlaku.}

\end{enumerate}

\subsubsection*{Krizové scénáře}
\begin{enumerate}[leftmargin=*]
\item Jako posunovač máte volnou chvíli, ohlédnete se do sousední stanice
a~vidíte, že dva vlaky jedou proti sobě a~nezpomalují, přibližně za 5 s~do sebe
narazí. Co uděláte?
\solution{Klik na červenou ikonku na horní liště panelu \textit{Zastavit DCC
na celém kolejišti}. Poté otevřít ovladač obou souprav, zastavit je a pustit
DCC. Tím lokomotivy zůstanou stát a~celé kolejiště se rozjede.}

\item Kdy použít nouzové zastavení celého kolejiště a~kdy vybrané soupravy?
\solution{Nouzové zastavení celého kolejiště používejte v~případě, kdy není
možné vlak zastavit jinak. Typicky například proto, že se vlak nenachází
v~obvodu vámi řízené stanice. Nouzové zastavení celého kolejiště je krajní
volbou!}

\end{enumerate}

\subsubsection*{Odpovědnost}
\begin{enumerate}[leftmargin=*]
\item Popište, za co jako dispečer odpovídáte.
\solution{Za všechny soupravy v~obvodu vámi řízených stanic – za to, že se
nepoškodí.}

\item Co dělat, když na kolejišti něco rozbijete?
\solution{Nahlásit starší obsluze.}

\end{enumerate}

\subsubsection*{Vlakotvorba}
\begin{enumerate}[leftmargin=*]
\item Jaké jsou základní režimy fungování kolejiště z~hlediska vlakotvorby
a oběhů vlaků?
\solution{Neježdění nákladní dopravy × ježdění nákladní dopravy. \\
Ježdění v~taktu × provoz podle grafikonu.}

\item Popište, podle čeho se rozhodujete, jaký vlak kdy a~kam poslat.
\solution{Průběžně sledujete stav v~sousedních stanicích a~snažíte se jim vyjít
vstříc. Pokud mají plno, neposíláte vlaky a~místo toho třeba posunujete. Pokud
se do sousední stanice chystá osobní vlak, pošlete taky osobní, abyste
modelovali přípoje atp. Ideální je si na začátku směny se sousedními stanicemi
dohodnout takt (např. \uv{jeden vlak za jeden vlak}), případně jezdit podle
grafikonu.}

\item Co to znamená, že se na kolejišti \textit{jezdí nákladní doprava} a~jak
nákladní doprava funguje?
\solution{Pokud se \textit{jezdí nákladní doprava}, každý vůz každého
manipulačního vlaku má přesně stanoveno, odkud a~kam jede. Např. \textit{první
vůz Ztr v~soupravě 804734 jede z~Čejče k~silu Uhřice}. Tyto informace jsou
zapsány v~poznámce soupravy. Obsluhy stanic se podle nich řídí. \\ Vůz je
považován za vyložený/naložený po průjezdu jednoho Mn vlaku: první Mn vlak vůz
přiveze, po druhý vlak zůstává vůz ve stanici, třetí Mn vlak vůz odváží.}

\item Kde se dozvíte, jaké zboží a~jakými vozy stanice vyváží a~dováží?
\solution{\textit{Platné pouze pro modulovku TT} \\ U~každé stanice je výtisk
datových listů. Dostupné také online:\\
\url{https://github.com/kmzbrnoI/station-datasheets/releases}}.

\item Které úkony je nutné vykonat k~řádnému zpracování manipulačního vlaku,
který vám právě přijel do stanice? Zpracování manipulačního vlaku končí
v~momentě jeho odjezdu ze stanice.
\solution{\textit{Relevantní pouze při ježdění nákladní dopravy.} \\ V~editaci
vlaku je v~poznámce k~soupravě uvedeno, jak naložit s~jednotlivými vagóny.
Vyřešte požadavky, které se týkají vaši stanice (připojení a~odpojení vozů).
Přidejte do soupravy vagóny dle poptávky ar~plánů. Upravte počet vozů, délku
vlaku a~popis vagónů dle změn.}

\item Potřebujete lokomotivní zálohu z~výtopny/depa, jak ji získáte?
\solution{Zavoláte/napíšete do depa a~zálohu si vyžádáte.}

\item Kam píšete nákladní vozy ve vlaku?
\solution{V~editaci vlaku do pole \textit{Poznámky} k~soupravě.}

\end{enumerate}

\subsubsection*{Další}
\begin{enumerate}[leftmargin=*]
\item Co je to riziková funkce?
\solution{Jedná se o~rizikové operace, jako například rušení vlaku,
přestavování obsazené výhybky, uvolňování závěru. Při průběhu rizikové funkce
je zobrazeno speciální okno, kde je třeba operaci ještě jednou potvrdit.}

\item Které všechny úkony je nutné vykonat při potvrzování rizikové funkce?
Uveďte na příkladu nouzového stavění výhybky.
\solution{Je nutné přečíst si potvrzovanou rizikovou funkci a výpis
kontrolovaných podmínek. Občas se ve výpisu může objevit čekání na nějakou
akci (uzavření přejezdu / přestavení výhybky), v~takovém případě je nutné
počkat na dokončení těchto úkonů. A~poté odsouhlasit seznam kontrolovaných
podmínek.}

\end{enumerate}


\newpage
\section{S1u – dispečer řízení jízdy přes uLI}

Musí být součástí zaškolení úrovně S1 pro kolejiště, kde se používají
uLI-daemon.

\subsection{Praktická část}

\begin{enumerate}[leftmargin=*]
\item Proveďte převzetí, řízení a uvolnění soupravy na Roco Multimaus.
\end{enumerate}

\subsection{Teoretická část}

\begin{enumerate}[leftmargin=*]

\item Co je to \textit{uLI-daemon} a~jakou má funkci?
\solution{Je to aplikace umožňující funkci uLI-master. Spouští se se startem
prvního panelu na dispečerském počítači, při nechtěném ukončení lze znovu
zapnout ikonkou na horní liště panelu.}

\item Co je to \textit{uLI-master} a~jakou má funkci?
\solution{Je to černá krabička rozměru 7×5 cm, umožňující připojení Roco
Multimaus k~počítači se spuštěným panelem a uLI-daemonem.}

\item Ikonka mašinky na Roco Multimaus bliká, co to znamená?
\solution{Multimaus neovládá žádnou lokomotivu.}

\item Ikonka mašinky na Roco Multimaus trvale svítí, mašinka přesto nereaguje,
co je špatně?
\solution{Může to znamenat, že není zaškrtnuto \textit{Ruční řízení} (tedy, že
ovládáte pouze funkce a~nikoliv jízdu) nebo lokomotiva/y uvízla na špíně nebo
jste na ovladač přidělili špatnou lokomotivu/y.}

\end{enumerate}


\newpage
\section{S2 – pokročilý dispečer}

Opravňuje k~řízení větších stanic. Obsahuje otázky na veškerou funkcionalitu,
kterou umí hJOP nabídnout.

\subsection{Praktická část}

\begin{enumerate}[leftmargin=*]
\item Demonstrujte zapnutí a~vypnutí kolejiště a~serveru.
\item Rozsviťte přivolávací návěst na návěstidle bez postavení nouzové cesty.
\item Proveďte příjem vlaku na obsazenou kolej včetně zrušení nouzové cesty.
\item Ručně uzavřete přejezd.
\item Proveďte nouzové otevření přejezdu.
\item Vytvořte a zrušte soupravu s~více lokomotivami.
\item Přidejte na kolej více souprav.
\item Přesuňte soupravu na jinou kolej.
\item Prohoďte soupravy na jedné koleji.
\item Proveďte vytvoření nové lokomotivy.
\item Proveďte úpravu dat lokomotivy.
\item Zjistěte, kde se nachází lokomotiva s~danou adresou.
\item Nastavte modelový čas, zapněte jej a vypněte jej.
\item Předveďte práci se zásobníkem povelů: přidání povelu, pozastavení
zásobníku, smazání povelu, přesun povelů.
\item Proveďte maximálně možně zabezpečený nouzový vjezd vlaku na manipulační
kolej.
\item Demonstrujte multitrakci v~ručním ovladači.
\item Zaveďte předvídaný odjezd vlaku.
\item Zaveďte na návěstidle režim AB, zrušte režim AB.
\item Postavte JC s~variantními body.
\item Postavte složenou JC s~variantními body.
\end{enumerate}

\newpage
\subsection{Teoretická část}

\begin{enumerate}[leftmargin=*]
\item Vysvětlete význam následujících symbolů. \\
\symbol{symboly/kol5.png}{0.1}{Kolej volná + nerozlišený závěr.}
\symbol{symboly/kol7.png}{0.1}{Volná + nerozlišený závěr + varovný štítek.}
\symbol{symboly/kol20.png}{0.1}{Volná + výluka.}
\symbol{symboly/kol21.png}{0.1}{Volná + varovný štítek.}

\symbol{symboly/hlnav9.png}{0.1}{Povolující návěst pro vlak (mimo PN) + volba
vlakové cesty.}
\symbol{symboly/hlnav15.png}{0.1}{Základní stav + volba nouzové cesty.}
\symbol{symboly/hlnav17.png}{0.1}{Povolující návěst pro vlak (mimo PN) +
aktivní režim AB.}
\symbol{symboly/hlnav18.png}{0.1}{Základní stav + aktivní režim AB.}
\symbol{symboly/hlnav19.png}{0.1}{Zhaslé/změna návěsti + aktivní režim AB.}
\symbol{symboly/senav14.png}{0.1}{Základní stav seřaďovacího návěstidla + volba
nouzové posunové cesty.}

\symbol{symboly/vyh7.png}{0.1}{Úsek obsazen + nouzový závěr výhybky.}
\symbol{symboly/vyh8.png}{0.1}{Úsek obsazen + nouzový závěr výhybky + ztráta
dohledu.}
\symbol{symboly/vyh14.png}{0.1}{Úsek volný + výhybka hlásí obě polohy.}
\symbol{symboly/vyh21.png}{0.1}{Závěr vlakové cesty + nouzový závěr výhybky +
úsek volný.}
\symbol{symboly/vyh22.png}{0.1}{Závěr posunové cesty + nouzový závěr + úsek
volný.}
\symbol{symboly/vyh23.png}{0.1}{Nerozlišený závěr jízdní cesty + úsek volný
(potenciálně nouzový závěr výhybky).}
\symbol{symboly/vyh24.png}{0.1}{Nerozlišený závěr jízdní cesty + úsek volný +
varovný štítek.}
\symbol{symboly/vyh25.png}{0.1}{Varovný štítek + úsek volný + ztráta dohledu.}
\symbol{symboly/vyh26.png}{0.1}{Výluka + nouzový závěr výhybky + úsek volný.}
\symbol{symboly/vyh27.png}{0.1}{Výluka + nouzový závěr výhybky + úsek volný +
ztráta dohledu.}

\symbol{symboly/vyk9.png}{0.1}{Nouzový závěr výkolejky + úsek volný.}
\symbol{symboly/vyk10.png}{0.1}{Nouzový závěr + úsek obsazen + varovný štítek.}
\symbol{symboly/vyk11.png}{0.1}{Nouzový závěr + úsek obsazen + varovný štítek +
ztráta dohledu.}
\symbol{symboly/vyk17.png}{0.1}{Nouzový závěr + úsek volný + výluka.}

\symbol{symboly/ez4.png}{0.05}{Klíč vyjmut + varovný štítek.}

\item Nouzové závěry kterých bloků se automaticky neruší při volbě \textit{RNZ
– rušení nouzových závěrů} a~je třeba je zrušit ručně?
\solution{Závěr trati, uzavření přejezdu, závěr zámků.}

\item Jaké všechny povely lze vkládat do zásobníku povelů?
\solution{Stavění jízdní cesty (vlakové i~posunové cesty), žádost o~traťový
souhlas, udělení traťového souhlasu.}

\item Zásobník povelů je přepnutý do volby \textit{PV}, je zásobník aktivní?
\solution{Ne. Dočasné přepnutí zásobníku do volby \textit{PV} se používá
k~dočasnému pozastavení funkce zásobníku, například při nutnosti aktuálně
vykonat prioritní povel.}

\item Mohu vypnout napájecí zdroj stanice za plného provozu? Co po vypnutí
napájení udělá ovládací panel?
\solution{Ano, je to možné, ale zastaví to provoz v~celém napájeném úseku.
Symboly na reliéfu zfialoví.}

\item Mohu při restartu hJOP nechat vlaky v~trati?
\solution{Ano, vlaky můžou zůstat kdekoliv. Při běžné vypínání kolejiště se
doporučuje dojet do stanic.}

\item Na jaké úseky lze a~na jaké nelze přesouvat vlaky?
\solution{Vlaky lze přesouvat na staniční koleje a~některé vlečky. Na zbylé
úseky (zhlaví,  záhlaví, trať, ...) soupravy přesouvat nelze.}

\item Vysvětlete barvy soupravy předvídaného odjezdu vlaku.
\solution{
Inverzní pozadí a popředí: odpočet času nezačal plynout. \\
Modré pozadí: do odjezdu zbývá více než 30 s. \\
Žluté pozadí: do odjezdu zbývá méně než 30 s.}

\item Vysvětlete chování režimu AB u~návěstidla.
\solution{Režim AB je spojený s~návěstidlem a~konkrétní jízdní cestou. Tento
režim opakovaně staví jízdní cestu po jejím projetí.}

\end{enumerate}

\end{document}
